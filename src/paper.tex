\documentclass[3p,times,procedia]{elsarticle}
\flushbottom

%% The `ecrc' package must be called to make the CRC functionality available
\usepackage{ecrc}
\usepackage[bookmarks=false]{hyperref}
    \hypersetup{colorlinks,
      linkcolor=blue,
      citecolor=blue,
      urlcolor=blue}
%\usepackage{amsmath}
\usepackage{graphicx}
\usepackage{caption}
\usepackage{subcaption}
\usepackage[utf8]{inputenc}

%% set the volume if you know. Otherwise `00'
\volume{00}

%% set the starting page if not 1
\firstpage{1}

%% Give the name of the journal
\journalname{Procedia Computer Science}

%% Give the author list to appear in the running head
%% Example \runauth{C.V. Radhakrishnan et al.}
\runauth{W. Charlton et al.}

%% Give the abbreviation of the Journal.
\jid{procs}

\usepackage{amssymb}

% if you have landscape tables
\usepackage[figuresright]{rotating}

% add words to TeX's hyphenation exception list
%\hyphenation{author another created financial paper re-commend-ed Post-Script}

% declarations for front matter

\begin{document}
\begin{frontmatter}

\dochead{The 9th International Workshop on Agent-based Mobility, Traffic and Transportation Models, \\ (ABMTRANS) March 23 - 26, 2021, Warsaw, Poland}%

\title{Open-Source Web-Based Visualizer for \\ Dynamic-Response Shared Taxi Simulations}

\author[a]{William Charlton}
\author[a]{Gregor Leich}
\author[a]{Ihab Kaddoura}
\author[a]{Kai Nagel}

\address[a]{Technische Universität Berlin, Chair of Transport Systems Planning and Transport Telematics, Straße des 17. Juni 135, 10623 Berlin, Germany}

%% ----- ABSTRACT -----
\begin{abstract}

We describe a unique, web-based data visualization portal developed for use by researchers and public transit agencies investigating future shared-taxi fleet scenarios. Augmenting or even replacing fixed-route transit lines with automated, connected, shared taxi fleets may be a desirable alternative in less-densely developed areas. The MATSim agent-based transport microsimulation model is used to study scenarios including status quo; dynamically-dispatched fleets with drivers; and fully autonomous fleets. This paper focuses on a data visualization portal which includes many interactive views, such as agent (taxi) movements color-coded by number of passengers and trip request origins and destinations, changes in roadway and passenger volumes compared to a base case, and more. The agent-based simulation covers a 24 hour simulation period; analysts can hone in on specific times of day to examine, e.g. school pickup/drop-offs or commute trips connecting to rail stations. The tool is in operation for several small cities and rural regions in Germany and was successfully used as an outreach tool in public meetings. In addition, developers of the MATSim DRT extension found the visualizations particularly useful for debugging both the algorithms and the scenario definitions. All code is open source and, while this specific study has a very specific use case, the portal has an extensible design that could be modified for other purposes.

\end{abstract}

\begin{keyword}
shared taxis; data visualization; matsim; public outreach
\end{keyword}


%\correspondingauthor[*]{Corresponding author. Tel.: +0-000-000-0000 ; fax: +0-000-000-0000.}
\cortext[cor1]{Corresponding author. Tel.: +49 30 314-23308 ; fax: +49 30 314-26269.}
\end{frontmatter}

\email{charlton@vsp.tu-berlin.de}

%% ----- INTRODUCTION -----
\section{Introduction}
\label{introduction}

Augmenting or replacing fixed-route transit lines with automated, connected, shared taxi fleets may be a desirable alternative for transit agencies in less-densely developed areas. The economics of fixed-route transit are difficult to square with sparse development patterns [*], and the advent of private ride-hailing services adds the "shared taxi" concept to people's travel options. The same technologies used by private ride-hailing, however, can also be modified and applied to future transit services.

In this study, a new dynamic-response shared taxi service, or "DRT", is examined as an alternative approach to serving the population of rural areas and less-dense portions of urban regions. Dynamic-response transit is analyzed using the MATSim agent-based transport microsimulation model. MATSim is an extensible, activity- and multi-agent based transport simulation, which enables the simulation of large scale scenarios. \cite{MATSimBook}. Crucially, MATSim simulates passenger transport in both private and public modes \cite{ZiemkeEtAl2019OpenBerlinScenario}.

The existing methodology behind using MATSim to simulate DRT is well-described by [*] and is not the focus of this paper. Rather, in the course of the effort to develop, analyze and compare various DRT scenarios, a need arose to create some truly unique data visualizations that could assist in improving and displaying the DRT scenarios themselves. A web-based portal is the result of this research. The portal includes many interactive views, such as agent (taxi) movements color-coded by number of passengers and trip request origins and destinations, changes in roadway and passenger volumes compared to a base case, and more. The agent-based simulation covers a 24 hour simulation period; analysts can hone in on specific times of day to examine, e.g. school pickup/drop-offs or commute trips connecting to rail stations.

The tool is now in operation for several small cities and rural regions in Germany, and was successfully used as an outreach tool in online public meetings. Project partners at transit agencies are interested in understanding the detailed implications of the DRT simulations, and without a visual, interactive component, that would be extremely difficult to provide.

Furthermore, developers of the MATSim DRT extension found the visualizations particularly useful for debugging both the algorithms and the scenario definitions. The latest implementions of MATSim include extensive DRT capabilities whose methods are continually being optimized and improved.

All code for MATSim and the visualization portal is open source and, while this specific study has a quite esoteric use case, the portal has an extensible design that could be modified for other purposes.

%% ----- EXISTING TOOLS AND RESEARCH -----
\section{Motivation: The State of Existing Tools}
\label{motivation}

- Existing tools are incapable of visualizing the outputs of the DRT module of MATSim in an efficient manner
- Authors of the DRT modules of MATSim expressed an unfulfilled need to debug the code
- Previous research on web-based MATSim visualization provides a proof-of-concept and extensible open-source platform on which to build DRT-specific visualizations; an obvious extension!
- Agent-based microsimulation is quite complex and the reams of data and extensive details are not appropriate for public outreach and stakeholder meetings. PDF reports and powerpoint slides are the norm, but there is an opportunity here to build something truly interactive, possibly an order of magnitude more useful.

%% ----- REQUIREMENTS -----
\section{Requirements}
\label{requirements}

- Open source code so others may build upon it and extend it
- Works in any modern browser. IE11 and small players, no. (pull from other doc)
- Must somehow manage massive output datasets from MATSim runs. Browsers can't load 800MB files support massive datasets; requires some sort of output post-processing and storage of results
- Visually compelling and easy to organize results for various scenarios
- Public facing scenarios and private debugging areas


%% ----- EXPERIMENTS -----
\section{The Visualization Platform}
\label{platform}

Platform

- MatHub platform had extensive back-end server infrastructure (six server images for user authentication, file storage, post-processing, real-time animations). But it sucked
- MatHub front-end used MapBox API, but size of datasets made all but the simplest of scenarios difficult or impossible to visualize within browser memory constraints.

- Covid-19 portal threw out everything except centralized file storage, and relied on the analyst to run standardized post-processing scripts and store those smaller datasets. First experiments with ThreeJS were successful but very time consuming to write
- deck.gl proved far more performant than MapBox, also open source, and easier to write custom "shaders" using deck.gl examples

DRT animation

- Initially just the vehicle animations
- Analysts requested color by occupancy
- origin/destination "flyovers"


%% ----- DRT VISUALIZATIONS -----
\section{Dynamic-Response Taxi Visualizations}
\label{drtviz}

{Strippgen2016}
Back-end consists only of file storage. Our research team has an existing "Subversion" server which handles large files efficiently and retains file version history. Other file storage solutions such as cloud-based AWS etc could also be implemented (but weren't) -- all file management goes through a single File API which could have multiple back-ends.

Front end code is a "single page app" that can be hosted on any web server platform. We are using Github Pages because it is free and fast.

Web portal front-end has a simple text-based configuration file which lists the scenarios to be publicly viewable
- Analyst runs agent-based model and post-processing scripts to create road/transit network and DRT animation files
- etc

%% ----- RESULTS -----
\section{Results}
\label{results}

- It works!
- Website
- Six public scenarios
- Gregor's findings
- Transit agency public outreach meetings


FROM GREGOR

Not sure whether this helps, but is what comes to my mind: The drt viz helped us understand better how the drt code works and find some issues, e.g. when the system is congested weird things can happen and a large group of vehicles with one passenger per vehicle only moves from the same start to the same destination at the same time (in an attempt to save each passenger a few seconds it serves them separately and wastes vehicles which then are missing for the next requests). It is really nice to have requests and vehicle occupancy shown, Via did only the latter after some pre-processing. Other than that the website obviously made the avoev workshops easier. Instead of only sharing our screen we could hand out a website where people could click themselves. Unfortunately it seemed that those listeners did not use the website a lot and rather kept listening to us. Some hypotheses: Maybe because sometimes a group of people shared one computer and their internet connection was bad, maybe because they are not used to that kind of interaction (maybe they always just listen to others' presentations unless they present themselve and discuss verbally later). For us the other plots like pax/link or veh/link are not entirely new, some used QGis to produce those. However maybe the website forced us to actually produce all those plots systematically instead of an "we could do that plot, but maybe it's not necessary, let's look into something else or raw data only to avoid the hassle of setting up QGis..."

%% ----- CONCLUSIONS -----
\section{Conclusions and Outlook}
\label{conclusions}

This is the platform we've been trying to build for three years. It's efficient, extensible, and open source.

Further enhancements to the underlying platform and to the DRT visualization capabilities are expected.

Plugin architecture means that other javascript developers can write plugins for their own agent-based models and their own use cases.

\section{Acknowledgements}
This research was funded in part by the German Federal Ministry of Transport and Digital Infrastructure (funding number 16AVF2160)

\bibliography{vsp,ref,book}
\bibliographystyle{elsarticle-harv}

% \begin{thebibliography}{1}

%   \bibitem{R1}
%   Horni, A., Nagel, K., \& Axhausen, K. W. (Eds.). (2016). \textit{The multi-agent transport simulation MATSim} (p. 618). London: Ubiquity Press.

%   \bibitem{R2}
%   Strippgen, David (2016) \textit{OTFVis: MATSim’s Open-Source Visualizer}. In Andreas Horni, Kai Nagel, Kay W. Axhausen (Eds.): The Multi-Agent Transport Simulation MATSim: Ubiquity Press, pp. 225-–234.

%   \bibitem{R3}
%   Rieser, Marcel (2016) \textit{Senozon Via}. In Andreas Horni, Kai Nagel, Kay W. Axhausen (Eds.): The Multi-Agent Transport Simulation MATSim: Ubiquity Press, pp. 219–-224.

%   \bibitem{R4}
%   Free Software Soundation (2007) \textit{GNU General Public License}, Version 3.
%   URL: www.gnu.org/licenses/gpl.html

%   \bibitem{R5}
%   Satyanarayan, A., Moritz, D., Wongsuphasawat, K., and Heer, J. (2016) \textit{Vega-Lite: A Grammar of Interactive Graphics}. IEEE Transactions on Visualization and Computer Graphics, Volume 23, Issue 1. DOI: doi.org/10.1109/TVCG.2016.2599030

%   \bibitem{R6}
%   Wilkinson, Leland (2005) \textit{The Grammar of Graphics}, Second Edition. Springer Press, Chicago, USA. DOI: doi.org/10.1007/0-387-28695-0

%   \end{thebibliography}

\end{document}
